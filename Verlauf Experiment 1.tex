\documentclass[ ngerman, fontsize= 12pt, paper=a4, headings=big, titlepage=true]{article}

\usepackage{babel}
\usepackage[utf8]{inputenc}
\usepackage[T1]{fontenc}

%\usepackage{lmodern, microtype}
\usepackage{csquotes}
\usepackage{hyperref}
\usepackage{enumerate}

%Tablellen:
\usepackage{multirow}
\usepackage{caption}
\usepackage{booktabs}
\usepackage{rotating}

%Mathematik
\usepackage{amsmath}
\usepackage{amsfonts}
\usepackage{amssymb}
\usepackage{mathtools}
\usepackage{bbm}
\usepackage{color}
\usepackage[
left=3cm,
right=2cm,
top=2.5cm,
bottom=2cm,
%includeheadfoot
]{geometry}

\begin{document}
	
\begin{center}
	
\textbf{\Large{ Übersicht über die Versuchssplanung der 1.Runde}} \\
	
Kaya Maria Bayer, Ketevan Gurtskaia, Alicia Hemmersbach, Danuscha Große-Hering

\end{center}

\textbf{1. Einleitung (ein Satz reicht bei Präsentationen und Postern)}
\begin{enumerate}[-]
	\item Motivation, kurze Beschreibung von Inhalt und Ziel des Experimentes
	\item kurze Erläuterung der Vorgehensweise
	\item Überblick über die einzelnen Kapitel
\end{enumerate}

\textbf{2. Problemstellung und Versuchsbedingungen}
\begin{enumerate}[-]
	\item Beschreibung der Ziele des Experimentes \\
		Wie genau können vorgegebene Längen gezeichnet werden? Können kürzere Längen besser gezeichnet werden?
	\item Beschreibung der Versuchsbedingungen\\
		Ort: möglichst leerer Raum, damit keine Längen verglichen werden können\\
		Längen: 5 cm und 20 cm\\
		Probanden: 32\\
\end{enumerate}

\textbf{3. Analyse des Problems}
\begin{enumerate}[-]
	\item Was ist die interessierende abhängige Variable?- vorgegebene Länge
	\item Was sind interessierende Einflussvariablen und wie können diese variiert werden? \\
		Größe des Papiers, Größe der Stiftmine
	\item Was sind mögliche Störvariablen und welche können kontrolliert werden?\\
		Sehschwäche (nicht kontrollierbar), Oberfläche des Tisches, Anzahl der Personen im Raum, Alter, Geschlecht 
	\item Von welchen Störvariablen soll noch der Einfluss erfasst werden? \\
		Größe des Papiers,  Größe der Stiftmine
	\item Welche Störvariabalen sollen als Blockvariablen aufgefasst werden? \\
		Größe des Papiers(DIN A4, DIN A3), Größe der Stiftmine(2mm, 0.5mm)
\end{enumerate}

\textbf{4. Modell, Hypothesen und statistische Auswertungsmethoden}
\begin{enumerate}[-]
	\item  Mathematische Formulierung des Modells und der Null- und Alternativ-Hypothesen \\
		$xi_n$ := Länge  der gezeichneten Linie, $Xi:=$ Vorgegebene Länge $\Rightarrow yi_n := xi_n-Xi$ \\
		Hypothese 1: $yi_n =0 $  vs $. yi_n \neq 0$ \\
		Hypothese 2: $x1_n \geq x2_n  $ vs. $ x1_n < x2_n$
	\item Nennung der statistischen Auswertungsmethoden inkl. der R-Funktionen. Die Darstellung soll möglichst allgemein sein und sich nicht auf den Spezialfall des Experimentes
	beziehen.
\end{enumerate}

\textbf{5. Versuchsplanung}
\begin{enumerate}[-]
	\item  Genaue Beschreibung der Versuchsplanung, d.h.
	\begin{enumerate}[*]
		\item Angabe, was für die Maximierung der Primärvariation, für die Minimierung der
		Sekundärvariation und für die Identifizierbarkeit dient.
		\item Verwendung von Fachbegriffen wie Eliminierung, Konstanthaltung, Art der Verblindung, Blockbildung, Parallelisierung, Randomisierung, Umwandlung von Störvariablen, Wiederholungsmessung
		\item Nutzen von vollständig randomisierten Plänen, randomisierten vollständigen Blockplänen, systematisch variierten vollständigen Blockplänen, geschachtelten Blockvariablen, balanzierten unvollständigen Blockplänen

		\item Geben Sie genau an, mit welchen R-Funktionen, was bestimmt wurde.
		\item Bei Bestimmung des Stichprobenumfanges soll die Vorgehensweise genau beschrieben
		werden. 
		\item Außerdem soll beschrieben werden, wer, wann, was, wie, unter welchen Bedingungen
		untersuchen soll. Dabei reicht im Hauptteil des Berichtes, in der Präsentation und
		im Poster die Angabe des Namens der Methode bzw. der R-Funktion, mit der
		der Versuchsplan und die Randomisierung gewonnen wurde. Die genaue Angabe,
		wer, wann, was, wie, unter welchen Bedingungen untersuchen soll, kann im Anhang
		gegeben werden
		
	\end{enumerate}
\end{enumerate}

\end{document}
