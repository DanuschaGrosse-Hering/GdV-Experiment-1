
\documentclass[ ngerman, fontsize= 12pt, headings=big, titlepage=true]{beamer}

\usepackage{babel}
\usepackage[utf8]{inputenc}
\usepackage[T1]{fontenc}

%\usepackage{lmodern, microtype}
\usepackage{csquotes}
\usepackage{hyperref}
\usepackage{enumerate}

%Tablellen:
\usepackage{multirow}
\usepackage{caption}
\usepackage{booktabs}
\usepackage{rotating}

%Mathematik
\usepackage{amsmath}
\usepackage{amsfonts}
\usepackage{amssymb}
\usepackage{mathtools}
\usepackage{bbm}
\usepackage{xcolor}

\usetheme{Dresden}
\usecolortheme{spruce}

\title{Grundlagen der Versuchsplanung-Experiment 1}
\subtitle{Längenschätzen}
\author{Kaya Maria Bayer \and Ketevan Gurtskaia \and Alicia Hemmersbach \and Danuscha Große-Hering}
\date{07.Juni 2021}
\begin{document}
\begin{frame}[plain]
    \maketitle
\end{frame}

\begin{frame}{Inhaltsverzeichnis}
	\tableofcontents
\end{frame}

\section{Einleitung}
\begin{frame}{Einleitung}
\end{frame}

\section{Problemstellung und Versuchsbedingungen}
\begin{frame}{Problemstellung und Versuchsbedingungen}
	
\end{frame}

\section{Analyse des Problems}
\begin{frame}{Analyse des Problems}
	Inhalt...
\end{frame}

\section{Modell, Hypothesen und statistische Auswertungsmethoden}
\begin{frame}{Modell, Hypothesen und statistische Auswertungsmethoden}
	Inhalt...
\end{frame}

\section{Versuchsplanung}
\begin{frame}{Versuchsplanung}
	Inhalt...
\end{frame}

\section{Literaturverzeichnis}
\begin{frame}{Literaturverzeichnis}
	Inhalt...
\end{frame}

\end{document}
